\chapter*{Заключение}
\addcontentsline{toc}{chapter}{Заключение}	%	Добавляем "Заключеник" в оглавление

В настоящее время игровая индустрия полна физическими движками разных типов. Но большинство из них жертвую скоростью 
для достижения лучшей управляемости и поддержки кода, либо, наоборот, жертвуют понятностью и доступностью для разработчиков,
но обеспечивая лучшую производительность.

Написание собственной физической библиотеки на языке программирования D по-прежнему остается актуальным,
язык позволяет писать быстрые, удобные и интуитивно понятные для разработчика приложения.

На основе изученных математических и физических законов была написана объемная библиотека математики,
включающая в себя реализацию таких математических объектов как \textit{кватернион, вектор, квадратная матрица}.

Был реализован простейший геометрический примитив --- сфер.

Исследованы особенности пересечения шаров и их контактов в трехмерном пространстве. 

С помощью математической библиотеки реализована физическая библиотека, позволяющая моделировать физическое поведение
сфер разной массы и диаметра.

Реализовано реалистичное поведение коллизии сфер: изменение направления движения, кручения сфер.

Реализована возможность создания как упругого так и абсолютно неупругого контакта сфер.

В качестве интегратора использовался интегратор Эйлера. Метод интегрирования Эйлера позволяет получить высокую скорость расчетов
при невеликой погрешности.

В заключение, стоит отметить, что физическая библиотека прошла успешное первое тестирование. Как и ожидалось шары вступают в контакт,
обладая разными скоростями, вращениями и массами, а затем реалистично разлетаются приобретая новые скорости и вращения.