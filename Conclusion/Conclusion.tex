\chapter*{Заключение}
\addcontentsline{toc}{chapter}{Заключение}	%	Добавляем "Заключеник" в оглавление

В настоящее время игровая индустрия полна физическими движками разных типов. Но большинство из них жертвуют скоростью 
для достижения лучшей управляемости и поддержки кода, либо, наоборот, жертвуют понятностью и доступностью для разработчиков,
но обеспечивая лучшую производительность.

Написание собственной физической библиотеки на языке программирования D по-прежнему остается актуальным,
язык позволяет писать быстрые, удобные и интуитивно понятные для разработчика приложения.

\begin{itemize}
 \item На основе изученных математических и физических законов была написана объемная библиотека математики,
включающая в себя реализацию таких математических объектов как \textit{кватернион, вектор, квадратная матрица}.
 \item Был реализован простейший геометрический примитив --- сфера.
 \item Исследованы особенности пересечения шаров и их контактов в трехмерном пространстве. 
 \item С помощью математической библиотеки реализована физическая библиотека, позволяющая моделировать физическое поведение
сфер разной массы и диаметра, упругости.
 \item Реализовано реалистичное поведение коллизии сфер: изменение направления движения, кручения сфер.
 \item Реализована возможность создания как упругого так и абсолютно неупругого контакта сфер.
 \item В качестве интегратора для моделирования движения использовался интегратор Эйлера. Метод интегрирования Эйлера позволяет получить высокую скорость расчетов
при допустимой погрешности.
\end{itemize}

Физическая библиотека прошла успешное первичное тестирование. Как и ожидалось шары вступают в контакт,
обладая разными скоростями, вращениями и массами, а затем реалистично разлетаются приобретая новые скорости и вращения.