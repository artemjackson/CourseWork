\chapter{Обнаружение столкновений}
\

Описание движения тела полностью абстрагировано от его формы
или, как принято выражаться, геометрии – это очень важно
уяснить для понимания внутренней архитектуры физического
движка. Реализации твердого тела и геометрического объекта
как правило раздельны – геометрия тела играет роль только в
процессе обнаружения столкновений, рассматриваемый ниже.
\\

Модули обнаружения столкновений (collision detection) и 
реагирования на столкновения (collision responce) представляют 
собой ключевые компоненты любого физического движка. Именно
они в процессе симуляции осуществляют главную нагрузку на
процессор, и именно от них требуется максимальная точность.
\\

Методы обнаружения столкновений подразделяются на
две категории: статические и динамические. В первом случае
движение тел рассматривается как последовательность
«квантовых скачков» – проверка на пересечение двух геометрий
производится в промежутках между скачками, как если бы
объекты в этом время не двигались (то есть, имели нулевые
скорости). Основным допущением этого метода является
временная когерентность системы – предполагается, что тела
обладают небольшими скоростями, и их позиции не меняются
слишком резко с течением времени. Иными словами,
статическое обнаружение столкновений плохо подходит для
моделирования, например, пушки, стреляющей в бетонную
стену: за шаг времени (dt) снаряд пройдет расстояние,
значительно превышающее толщину стены, и, следовательно,
попросту пролетит сквозь нее – статическая проверка это
столкновение не обнаружит.

Динамическое обнаружение столкновений использует
другой метод, при котором учитывается скорость тела – движок
предсказывает, какое положение в пространстве займет объект
на следующем шаге времени, если будет двигаться с той же
скоростью. Фактически, делается проверка на пересечение
отрезка, представляющего собой траекторию движения тела, с
препятствием – вычисляется точка, в которой тело столкнется с
этим препятствием, и на основании этой информации движок
делает соответствующие корректировки. К сожалению,
динамический метод довольно сложен (особенно для
нетривиальных геометрий) и, в целом, плохо адаптирован для
динамики как таковой – он чаще используется в кинематике.
Поэтому далее будет рассматриваться статический метода – он,
при всех его недостатках, хорошо себя оправдывает в
большинстве физических ситуаций.
\\

В физическом движке функция проверки столкновения
между двумя объектами должна давать на выходе следующую
информацию:
\begin{itemize}
  \item точка столкновения
  \item нормаль к поверхности столкновения в этой точке
  \item глубина взаимного проникновения объектов
\end{itemize}

Эти три свойства, вкупе со ссылками на столкнувшиеся
тела, дают новую сущность – контакт. Соответственно,
процесс обнаружения столкновений в терминологии
физического движка также называют генерированием
контактов (contact generation). В сложных движках на одно
столкновение между двумя телами приходится несколько
контактов – так называемый contact manifold. Далее будет 
рассматриваться простейший случай – столкновение сфер, где будет
достаточно одного контакта на пару столкнувшихся тел.

