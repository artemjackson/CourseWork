\chapter*{Введение}	%	Пишем "Введение" вместо  "Глава 1 Введение"
\addcontentsline{toc}{chapter}{Введение}	%	Добавляем "Введение" в оглавление
Физический движок (англ. $physics$ $engine$) --- компьютерная программа, которая производит
компьютерное моделирование физических законов реального мира в виртуальном мире,
с той или иной степенью аппроксимации. Чаще всего физические движки используются не как отдельные 
самостоятельные программные продукты, а как составные компоненты (подпрограммы) других программ.

Все физические движки условно делятся на два типа: игровые и научные.
Первый тип используется в компьютерных играх как компонент игрового движка. В этом случае он должен
работать в режиме реального времени, то есть воспроизводить физические процессы в игре с той же самой скоростью,
в которой они происходят в реальном мире. Вместе с тем от игрового физического движка не требуется точности вычислений.
Главное требование — визуальная реалистичность, и для его достижения не обязательно проводить точную симуляцию
Поэтому в играх используются очень сильные аппроксимации, приближенные модели и другие релаксации существующей задачи.

Научные физические движки используются в научно-исследовательских расчетах и симуляциях, где крайне важна
именно физическая точность вычислений. Вместе с тем скорость вычислений не играет существенной роли.
Современные физические движки симулируют не все физические законы реального мира, а лишь некоторые,
причем с течением времени и прогресса в области информационных технологий и вычислительной техники
список "поддерживаемых" законов увеличивается. Так, например, на начало 2010 года физические движки 
могут симулировать следующие физические явления и состояния:
\begin{itemize}
  \item динамика абсолютно твердого тела,
  \item динамика деформируемого тела,
  \item динамика жидкостей,
  \item поведение газа, 
  \item поведение тканей,
  \item поведение веревок(тросы, канаты и т.д.).
\end{itemize}

Целью данной работы является написание библиотеки, моделирующей поведение твердых тел в трехмерном пространстве. 
Такая библиотека позволит разработчикам создавать приложения на основе физики твердого тела как в трехмерном, так и 
в двумерном пространстве. Такими приложениями могут является компьютерный игры.

Для достижение поставленной цели необходимо решить следующие задачи:
\begin{itemize}
  \item Рассмотреть и изучить законы изменения импульса, скорости и ускорения тела для моделирования
  движения тела в пространстве, законы изменения тензора инерции, угловой скорости и углового ускорения 
  для моделирования вращения тела в пространстве вокруг оси вращения.

  \item Рассмотреть особенности пересечений элементарных геометрических тел --- поскольку любое физическое тело обладает 
  геометрической формой, которую нельзя игнорировать при моделировании физических процессов.

  \item Рассмотрены основные способы интегрирования дифференциальных уравнений в рамках дисциплины численных методов, 
  позволяющие определить основные физические характеристики тела в момент времени.
\end{itemize}


В качестве референтного языка используется язык программирования D.
Данный язык является из семейства языков программирования "фигурных скобок"
поэтому представленный в приложениях код будет интуитивно понятен читателю,
владеющему знаниями языка программирования С/С++ или Java.

Язык программирования D был выбран в связи с мощной реализацией шаблонов,
объемной стандартной библиотекой, а так же с быстротой работы.
Язык программирования D имеет широкое распространение от программного обеспечения веб-серверов
до бизнес решений и интернет приложений.