\chapter*{Введение}
\

Казалось бы, кому в наше время может прийти в голову
идея создания собственного физического движка? Это далеко не
тривиальная задача, которую давно уже решили за вас
разработчики $Bullet$, $PhysX$, $ODE$, $Newton$ и других известных
библиотек симуляции динамики твердых тел. А дело вот в чем.
\\

Поскольку физика – это достаточно сложная предметная
область, требующая специальных знаний, к этим библиотекам
традиционно сложилось отношение как к «черным ящикам».

Пользователь не знает и не пытается понять все хитроумные
алгоритмы и тонкости внутренней архитектуры физического
движка, он лишь использует его внешний API. При этом он
может прекрасно разбираться в 3D-графике и линейной
алгебре, но физика для него так и останется «terra incognita»,
если он будет полагаться на готовые решения.
Знать физику с точки зрения пользователя физического
движка и с точки зрения его разработчика – это две совершенно
разные вещи.
\\

Изучение физики поможет нам лучше понять то,
как работает практически любой физический движок, перестать
относиться к физике как к некой «магии», научиться
использовать готовые решения более грамотно и эффективно.

К тому же, абсолютное большинство движков перегружены
функциональностью и представляют собой, фактически,
комбайны с собственной системой проверки столкновении,
громадной математической библиотекой (векторы, матрицы,
кватернионы), реализацией геометрических тел и еще Бог знает
чего еще. 
\\

Во многих случаях использование сторонней
физики как минимум неоправданно – например, если вы
создаете бильярд или боулинг, то вряд ли вам понадобятся
архисложный солвер, поддержка всех мыслимых типов
сочленений и неконвексной геометрии...

Тем не менее, даже простой физический движок
реализовать весьма нелегко. В этой работе мы решили исправить эту
ситуацию и раскрыть ключевые моменты создания игровой физики.
\\

Главной задачей данной работы является написание библиотеки,
позволяющей разработчикам создавать приложения на основе трехмерной
физики твердого тела, такими могут является программы симулирующие реальные 
физические явления либо компьютерный игры.

Наряду с этим будет реализована поддержка двухмерной физики твердых тел.
Т.е. написанная библиотека будет удовлетворять требованиям целой ниши 
разработчиков компьютерной физики.

Для решения данной задачи понадобится изучить законы изменения
импульса, скорости и ускорения тела для реализации движения тела в пространстве,
законы изменения тензора инерции, угловой скорости и углового ускорения 
для реализации вращения тела в пространстве вокруг оси вращения.

Так же понадобится просчет пересечений элементарных 
геометрических тел - поскольку любое физическое тело обладает
геометрической формой, которую нельзя игнорировать при имитации 
физических процессов.

И в заключение, будут рассмотрены основные способы интегрирования
дифференциальных уравнений в рамках дисциплины численных методов,
позволяющие разрешить коллизию двух твердых тел.
\\

Основними литературными источникам являются свободнораспространяющийся
электронный журнал 'FPS', а так же сайт разработчиков физики и игр 'GameDev.ru'
\\

В качестве референтного языка используется язык программирования D.
Данный язык является из семейства языков программирования `фигурных скобок`
поэтому представленный в приложениях код будет интуитивно понятен читателю,
владеющему знаниями языка программирования С/С++ или Java.
Язык программирования D был выбран в связи с мощной реализацией шаблонов,
объемной стандартной библиотекой, а так же с быстротой работы.
Язык программирования D имеет широкое распространение от программного обеспечения и веб-серверов
до бизнес решений и интернет приложений