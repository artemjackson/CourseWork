\chapter*{Введение}

%TODO уводзіны не нумеруюцца (уставіў зорачку уверсе), і на падпункты не падзяляюцца
\section{Почему актуально написание собственного физического \\ движка?} 
\

Казалось бы, кому в наше время может прийти в голову
идея создания собственного физического движка? Это далеко не
тривиальная задача, которую давно уже решили за вас
разработчики $Bullet$, $PhysX$, $ODE$, $Newton$ и других известных
библиотек симуляции динамики твердых тел. А дело вот в чем.
\\

Поскольку физика – это достаточно сложная предметная
область, требующая специальных знаний, к этим библиотекам
традиционно сложилось отношение как к «черным ящикам».

Пользователь не знает и не пытается понять все хитроумные
алгоритмы и тонкости внутренней архитектуры физического
движка, он лишь использует его внешний API. При этом он
может прекрасно разбираться в 3D-графике и линейной
алгебре, но физика для него так и останется «terra incognita»,
если он будет полагаться на готовые решения.
Знать физику с точки зрения пользователя физического
движка и с точки зрения его разработчика – это две совершенно
разные вещи.
\\

Изучение физики поможет нам лучше понять то,
как работает практически любой физический движок, перестать
относиться к физике как к некой «магии», научиться
использовать готовые решения более грамотно и эффективно.

К тому же, абсолютное большинство движков перегру-
жены функциональностью и представляют собой, фактически,
комбайны с собственной системой проверки столкновении,
громадной математической библиотекой (векторы, матрицы,
кватернионы), реализацией геометрических тел и еще Бог знает
чего еще. 

\section{Почему использование сторонней физики не оправданно?} 
\

Во многих случаях использование сторонней
физики как минимум неоправданно – например, если вы
создаете бильярд или боулинг, то вряд ли вам понадобятся
архисложный солвер, поддержка всех мыслимых типов
сочленений и неконвексной геометрии...

Тем не менее, даже простой физический движок
реализовать весьма нелегко. В этой работе мы решили исправить эту
ситуацию и раскрыть ключевые моменты создания игровой физики.

\section{Цели и задачи исследования} 
\

В качестве главной задачи данного исследования, мы ставим цель
изучения законов физики твёрдого тела,  именно:

\begin{enumerate}
\item изучение законов изменения импульса, скорости и ускорения; \\
\item изучение законов изменения тензора инерции, угловой скорости
и углового ускорения; \\ 
\item изучение законов изменения позиции твёрдого тела в пространстве
и времени, а также изменение иего ориентации; \\
\item изучение законов и правил взаимодействия двух твёрдых тел; \\
\end{enumerate}

Сопутствующей целью будет просчет пересечений элементарных 
геометрических тел - поскольку любое физическое тело обладает
геометрической формой, которую нельзя игнорировать при имитации 
физических процессов.

Так же будут рассмотрены основные способы интегрирования
дефференциальных уравнений в рамках дисцеплины цисленных методов.

\section{Основные летературные и информационные источники.}
\

Поскольку большинство физических движков являются проприетарными
и совершенно закрытыми, найти хорошую литературу, а тем более
на русском языке, не представляется возможным. В связи с этим основным
источником информации стала всемирная сеть Интернет. Большая часть 
инормации, представленной в данной работе, была получина путем изучения
и анализировнаия форумов и электронных журналов, посвященных
``игростроению``и игровой физике.

Из них следует отметить интернет-форум разработчиков игр gamedev.ru,
а так же свободнораспространяющийся электронный журнал ``FPS``.
\\

В качестве референсного языка используется язык программирования D,
на момент написания данной работы, язык входил в 20 популярнейших языков
программирования. Несмотря на то, что данный язык программирования 
малоизвестен в широких кругах, представленный код будет понятен аналогичным 
для других родственных языков из семейства ``фигурных скобок``.


