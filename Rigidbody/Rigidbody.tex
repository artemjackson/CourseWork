\chapter{Твердое тело. Основные физические характеристики твердого тела}
\
\section{Масса и сила}
\

Для описания динамики тела используются две основные
физические величины: масса (m) и сила (F). В ньютоновской
механике масса характеризует инертность тела (способность
тела сопротивляться приложенной силе). В современной
физике понятие массы имеет несколько иной смысл (в
частности, в СТО масса движущегося тела – относительная
величина и зависит от его скорости), но будем
придерживаться «классического» определения.
\\

Масса – скалярная величина (выражена одним вещественным числом).
По системе СИ масса измеряется в килограммах (кг).
\\

Сила – мера интенсивности взаимодействия тел. Сила
является причиной изменения скорости тела. Сила является
векторной величиной, так как имеет направление (для трехмерного
пространства она выражается трехкомпонентным вектором). 
По системе СИ сила измеряется в ньютонах (Н).
\\

\section{Движение тела в пространстве. Скорость и ускорение}
\

Для того, чтобы описать движение тела, вводятся также
понятия скорости (v) и ускорения (a). По системе СИ скорость
измеряется в метрах в секунду (м/с), ускорение – в метрах в
секунду в квадрате (м/с2). Зависимость ускорения тела от его
массы и суммы приложенных к нему сил описывает второй
закон Ньютона:
\begin{equation}
a = \frac{F}{m} 
\end{equation}

Ускорение – это не что иное, как скорость изменения
скорости. Другими словами, ускорение показывает, на сколько
изменится вектор скорости за интервал времени (\begin{math}\Delta{t}\end{math}).

Формальное определение звучит так: ускорение – это
первая производная скорости по времени (t):
\begin{equation}
v(t) = v_0 + \int_{t_0}^t a(t)dt
\end{equation}

Необходимо пояснить: и скорость, и ускорение в
данном случае рассматриваются как функции. Функция,
определяющая скорость, изначально неизвестна. Однако,
известна скорость изменения этой функции в заданной точке (в
заданный момент времени) – ускорение. Функция скорости
изменения функции – это и есть производная функции.
Нахождение производной называется дифференцированием, а
обратный процесс (нахождение первообразной), который и
интересует – интегрированием.
\\

Если ускорение не меняется со временем, то
интегрирование скорости осуществляется по следующей
формуле:
\begin{equation}
v = v_0 + a\Delta{t} 
\end{equation}

Данный метод интегрирования называется методом
Эйлера.

\section{Позиция тела в пространстве}
\

Аналогично приведенному выше способу интегрируется и позиция тела – координаты
центра масс (x):
\begin{equation}
x = x_0 + v\Delta{t}
\end{equation}

Таким образом, динамическое тело может быть описано
следующими минимальными характеристиками:
\begin{itemize}
  \item масса (скаляр)
  \item позиция (вектор)
  \item скорость (вектор)
  \item ускорение (вектор)
  \item сумма сил (вектор)
\end{itemize}

\section{Тензор инерции. Вращательная сила}
\

В абсолютном большинстве случаев, помимо линейного (поступательного)
движения тел, интересует еще и вращательное. Концепции силы, скорости
и ускорения во вращательном движении похожи на их линейные
аналоги – с той разницей, что вместо массы в уравнениях
вращения используется момент инерции(I).
\

В реалистичных симуляциях это тензор (то есть, матрица \begin{math}3\times3\end{math}) – это мера инертности во
вращательном движении тела вокруг оси. По системе СИ
измеряется в килограмм-метрах в квадрате (кг·м2). Например, для сферы
момент инерции равен
\begin{equation}
I = \frac{2}{5} mr^2 
\end{equation}

Зная эту величину, по второму закону Ньютона можно 
вычислить угловое ускорение тела (\begin{math}\alpha\end{math}):
\begin{equation}
\alpha = IM
\end{equation}

где M – момент силы (вращающая сила). В системе СИ
измеряется в ньютон-метрах (H·м). Вращательное действие
линейной силы определяется как векторное произведение
силы на точку, в которой эта сила приложена:
\begin{equation}
M = r\times{F} 
\end{equation}

Исходя из углового ускорения, интегрируется угловая
скорость (\begin{math}\omega\end{math}, рад/с):
\begin{equation}
\omega = \omega_0 + \alpha{dt}
\end{equation}

Если угловые скорость и ускорение можно описать трехкомпонентными
векторами, то накапливаемый поворот из-за проблем с блокировкой оси
векторами обычно не описывают. Вместо них используются
матрицы \begin{math}3\times3\end{math}, либо кватернионы. Последние более выгодны, так
как требуют всего 4 вещественных значения (против 9 для
матриц).

Интегрирование кватерниона поворота (q) из угловой
скорости выглядит так:
\begin{equation}
q = \frac{1}{2}q_0\omega{dt}
\end{equation}

Итак, для того, чтобы учесть вращательное движение, мы
добавляем к нашему описанию тела еще пять характеристик:

\begin{itemize}
  \item момент инерции (скаляр)
  \item поворот (кватернион)
  \item угловая скорость (вектор)
  \item угловое ускорение (вектор)
  \itemсумма моментов сил (вектор)
\end{itemize}
\

В любой момент времени скорость тела зависит от
времени, прошедшего между двумя шагами симуляции – \begin{math}\Delta\end{math}t. Это
время, как правило, фиксировано и должно соответствовать
времени между двумя кадрами рендеринга – например, 1/60 с.

Это соответствует 60 кадрам в секунду, как в системах с
включенной вертикальной синхронизацией.
В кинематике (идеализированном описании движения) в
качестве \begin{math}\alpha\end{math} часто берется реальное время между кадрами, но
динамика такой подход «не любит»: при резких скачках \begin{math}\alpha\end{math}
(например, вследствие запоздания графики), тела
приобретают неестественно большие скорости и начинают
пролетать сквозь препятствия.
