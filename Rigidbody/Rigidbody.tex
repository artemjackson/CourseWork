\chapter{Моделирование динамики твёрдого тела}
\section{Типичные задачи для моделирования в компьютерных играх}

Компьютерные игры моделирует определённую окружающую среду,
называемую также игровым миром, в котором игровые объекты подчиняются
некоторым законам так, чтобы их поведение было ожидаемым и предсказуемым
для игрока. Для большинства игр основной интерес представляют передвижение
различных тел в пространстве, а также ряд визуальных феноменов механического
плана. Примерами таких игр являются: %TODO привести примеры игр
. Существуют игры из жанра головоломок (например, серия ``The incrdible machines'', ``Auditorium'' %TODO ещё примеры
), которые моделируют не только механические явления, однако они составляют
сравнительно малую часть всех игр.

Та часть компьютерной программы, которая отвечает за моделирования
различных физических явлений в игровом мире получила устойчивое
название physics engine --- физический движок (этим жаргонизмом ставшим термином мы и
будем пользоваться в дальнейшем). Как было рассмотрено выше физический
движок моделирует в основном механические явления. Наибольшее распространение и 
значимость для компьютерных игр имеют следующие задачи:
\begin{itemize}
 \item моделирование движения в поле консервативных сил (например, в гравитационном),
 \item моделирование столкновений тел (здесь рассматриваются как абсолютно упругие так и эластичные тела),
 \item моделирование сочленений и соединений,
 \item моделирование потоков воды, дыма, огня, которое обычно осуществляется системой частиц,
 \item моделирование тканей.
\end{itemize}
%TODO здесь возможно стоит подробно пройтись по каждой задаче, а также оговорить, что реализовано, а что только будет реализовано

%TODO нужно расказать об особенностях игровой физики: большое число объектов, realtime, визуальное, а не физическое подобие. 

\section{Физико-математическое описание механического движения}
Конечного пользователя интересует визуальное поведение игровых объектов.
Оно определяется положением тел в пространстве и времени и изучается кинематикой. Однако
кинематика не рассматривает причины возникновения движения --- этим занимается динамика. Поэтому
моделирование игровой сцены строится в следующем порядке: моделируется динамика (т.е. взаимодействие тел), на
его основе строится кинематическое описание, которое непосредственно и видит игрок. Фактически
моделирование игровой физики сводится к решению соответствующей обратной задачи механики.

% TODO добавить абзац о различных способах аналитического моделирование одних и тех же явлений: 
% силовая динамика, законы сохранения, принцип наименьшего действия
% Обосновать необходимость численного моделирования

% TODO добавить абзац о материальной точке, абсолютно твёрдом теле и упругом теле

\subsection{Кинематическое описание}
\subsubsection{Материальная точка}
Положение тела в пространстве в любой момент времени задаётся функцией радиус вектора от
времени, получившей исторически название уравнение движения:
\begin{equation}
 \mathbf{r}=\mathbf{r}(t).
\end{equation}

Непосредственный вид уравнения движения может быть задан лишь в ограниченном числе 
тривиальных случаев. В реальных же задачах мы можем получить на основе динамики 
дифференциальное уравнение (или систему в случае многих тел), решением которого
будет искомое уравнение движения.
\begin{equation}
 \ddot{\mathbf{r}}=f(t,\mathbf{r}(t),\dot{\mathbf{r}}(t)).
\end{equation}

Первой и второй производной является скорость и ускорение соответственно.
Так как механическое состояние системы полностью характеризуется заданием координат
и скоростей \cite[с. 10]{Landau}, то возможны уравнения не выше второго порядка. 
\subsubsection{Твёрдое тело}
Лишь некоторые тела могут промоделированы заданием лишь одного положения в пространстве 
(т.е. как материальная точка), большинство тел имеет форму и по разному ведут себя в зависимости
от ориентации в пространстве. Существует несколько способов задания ориентации тела в трёхмерном
пространстве. Изменение положения тела в пространстве характеризуются двумя псевдовекторами ---
угловой скоростью и угловым ускорением.
\subsubsection{Углы Эйлера}
%TODO здесь толком описать углы Эйлера (желательно с картинками) и проблему шарнирного замка
\subsubsection{Матрица поворота}
%TODO описать матрицу поворота
\subsubsection{Кватернион}
%TODO описать кватернион и его связь с поворотом и угловой скоростью
\subsection{Динамическое описание}
Согласно силовой модели механики равнодействующая сил приложенных к телу вызывает ускорение (второй закон Ньютона)
\begin{equation}
 \mathbf{a}=\frac{\sum{\mathbf{F}_i}}{m}.
\end{equation}
Всего существует только четыре природы силовых взаимодействий: гравитационная, электромагнитная, сильная и слабая.
Последние две проявляют себя только на микроуровне, поэтому их нет необходимости рассматривать в задачах
игровой физики. Гравитационная природа представлена полем тяготения, которое является консервативным. 
Все остальные силы являются проявлением взаимодействий электромагнитной природы на микро и макроуровне.
Однако оказывается весьма неудобным при рассмотрении макроявлений использовать электромагнитные силы на микроуровне,
поэтому вводятся макрообощения характеризующее соответствующее явление в целом, например, закон Кулона
для трения или закон Гука для упругости.

Во многих случаях даже такое макроописание является слишком подробным и сложно реализуемым на компьютере,
поэтому часто моделируют только абсолютно твёрдые тела, для взаимодействия которых выполняется закон 
сохранения импульса.
\subsubsection{Динамика вращательного движения}
Для вращательного движения тела можно ввести ряд характеристик
являющимися аналогами уже рассмотренных для поступательного движения.
Радиус вектору $ \mathbf{r} $ соответствует кватернион $q$ или матрица вращения $\mathrm{R}$,
скорости $\mathbf{v}$ --- угловая скорость $\boldsymbol{\omega}$ направленная вдоль оси вращения,
ускорению $\mathbf{a}$ --- $\boldsymbol{\alpha}$ угловое ускорение.
Вызванное силой вращение зависит не только от силы, но и от положения точки приложения силы
по отношению к мгновенному центру вращения, величина аналогичная силе носит название момента силы:
\begin{equation}
 \mathbf{M} = \mathbf{r}\times{\mathbf{F}}, 
\end{equation}
она также как и сила является аддитивной величиной.

Мера отклика на момент силы зависит не только от массы тела, но и от геометрии распределения
масс в теле. Данная величина получила название момента инерции тела:
\begin{equation}
 J=\int\limits_{V} \rho r^2\mathrm{d}V.
\end{equation}

В общем случае момент инерции зависит от направления оси вращения и поэтому является тензорной
величиной $\mathrm{J}$ (фактически двухмерной матрицей \begin{math}3\times3\end{math}), компоненты которого могут
быть выражены как:
\begin{equation}
 \mathrm{J}_{ij} = \int\limits_{V} (\delta_{ij}r^2 - \mathbf{r}_i \mathbf{r}_j) \rho \mathrm{d}V,
\end{equation}
где $\delta_{ij}$ символ Кронекера, а $\mathbf{r}$ радиус-вектор из центра вращения к точке интегрирования.
Тензор инерции всегда является положительно определённым и может быть приведён к диагональному виду. Момент
инерции относительно произвольно расположенной оси вращения может быть выражен через момент инерции параллельной
оси вращения проходящей через центр масс по теореме Гюйгенса--Штейнера:
\begin{equation}
 J=J_C+md^2.
\end{equation}

%TODO Дать формулу связывающую момент силы, момент инерции и угловое ускорение в случае, 
% когда момент инерции является тензорной величиной. (я пока нашёл лишь через кососимметричные матрицы) 
% также в обязательном порядке дать корректные формулировки связи кватерниона и угловой скорости
%\begin{equation}
%q = \frac{1}{2}q_0\omega{dt} % здесь наверное дэльта t, а не dt иначе у меня что-то не стыкуется
%\end{equation}

% Возможно нужно что-то ещё, но собственно это резюме этой части
Таким образом, динамическое тело может быть описано
следующими минимальными характеристиками:
\begin{itemize}
  \item масса (скаляр)
  \item позиция (вектор)
  \item скорость (вектор)
  \item ускорение (вектор) %???
  \item сумма сил (вектор) %???
  \item момент инерции (матрица)
  \item поворот (кватернион)
  \item угловая скорость (вектор)
  \item угловое ускорение (вектор) %???
  \item сумма моментов сил (вектор) %???
\end{itemize}
% величины отмеченные вопросом на мой взгляд являются вычислимыми или внешними величинами
% поэтому нужно ли их хранить в объекте (это касается не только текста, но и вашего кода).

% TODO это надо переформулировать и обосновать (что за ``не любит'', у этого есть математические обоснования)
% также нужно подумать где это расположить
%
%В любой момент времени скорость тела зависит от
%времени, прошедшего между двумя шагами симуляции – \begin{math}\Delta\end{math}t. Это
%время, как правило, фиксировано и должно соответствовать
%времени между двумя кадрами рендеринга – например, 1/60 с.
%
%Это соответствует 60 кадрам в секунду, как в системах с
%включенной вертикальной синхронизацией.
%В кинематике (идеализированном описании движения) в
%качестве \begin{math}\alpha\end{math} часто берется реальное время между кадрами, но
%динамика такой подход «не любит»: при резких скачках \begin{math}\alpha\end{math}
%(например, вследствие запоздания графики), тела
%приобретают неестественно большие скорости и начинают
%пролетать сквозь препятствия.
