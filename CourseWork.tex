\documentclass[a4paper,16pt]{report}	% Размер бумаги устанавливаем А4, шрифт 12пунктов
\usepackage[T2A]{fontenc}	% Поддержка русских букв
\usepackage[utf8]{inputenc}	% Включаем свою кодировку: koi8-r или utf8 в UNIX, cp1251 в Windows
\usepackage[english,russian]{babel}	% Используем русский и английский языки с переносами
\usepackage{amssymb,amsfonts,amsmath,mathtext,cite,enumerate,float}	% Подключаем нужные пакеты расширений
\usepackage[dvips]{graphicx}	% Хотим вставлять в диплом рисунки?
\graphicspath{{images/}}	% Путь к рисункам


\usepackage{enumitem}	
\setlist{nolistsep}	% отступы между элементами перечесления


\makeatletter
\renewcommand{\@biblabel}[1]{#1.}	% Заменяем библиографию с квадратных скобок на точку:
%\renewcommand{\baselinestretch}{1.65}.	% Полуторный интервал в тексте
\makeatother

\usepackage{geometry} 	% Меняем поля страницы:
\geometry{left=2cm}	% левое поле
\geometry{right=1.5cm}	% правое поле
\geometry{top=1cm}	% верхнее поле
\geometry{bottom=2cm}	% нижнее поле

\begin{document}
\begin{titlepage}
\newpage

\begin{center}
\textsc{\textbf{МИНИСТЕРСТВО ОБРАЗОВАНИЯ РЕСПУБЛИКИ БЕЛАРУСЬ}} \\
\vspace{1cm}
БЕЛОРУССКИЙ ГОСУДАРСТВЕННЫЙ УНИВЕРСИТЕТ \\
\end{center}
 
\vspace{8em}

\begin{center}
\Large \textsc{\textbf{Курсовая работа}} \\
по дисциплине  «Методы программирования» \\
на тему:
\end{center}

\vspace{2.5em}
 
\begin{center}
\textsc{\textbf{«Реализация трехмерной физики твёрдого тела \linebreak на языке программирования D»}}
\end{center}

\vspace{12em}
 
\begin{flushleft}
Выполнил студент 2-го курса \\
кафедры информационных технологий: \\
Степанищев ~А.~Э. \\
\vspace{1.5em}
Научный руководитель \\
Станкевич ~А.~А.\\
\vspace{1.5em}
\end{flushleft}
 
\vspace{\fill}

\begin{center}
Минск 2014
\end{center}

\end{titlepage}	% Титульный лист
\tableofcontents 	% Оглавление, которое генерируется автоматически
\chapter*{Введение}

%TODO уводзіны не нумеруюцца (уставіў зорачку уверсе), і на падпункты не падзяляюцца
\section{Почему актуально написание собственного физического \\ движка?} 
\ %як разумею гэтыя слэшы вы устаўлялі, каб атрымаць водступ у першым абзацы,
% але гэта не слушны падыход, прыблізнае тое ж, як народ на вардзе робіць абзацныя водступы прабеламі.
% Слушнае рашэнне, гэта наладзіць адпаведным чынам стыль усяго дакумента (я на гэтых днях, буду 
% актыўна гэтым займацца)

%У мяне у чора было мала часу, таму ранейшыя заўвагі былі толькі кветачкі, а зараз пойдуць ягадкі

%Так пісаць нельга. Ніякіх эмоцый ніякай, рыторыкі. Наколькі я вас
% зразумеў, думку якую вы жадаеце у гэтых трох абзацах выказаць,
% гэта тое, што не гледзечы на існаванне гатовых рашэнняў,
% карыстальнік недастакова ўяўляе што скрываецца за фізічным двіжком
% і ў сувязі з гэтым неслушна ім карыстаецца.
%
% У курсавой працы двэ асноўныя архімэты: вучэбная і даследчая, пры гэтым
% чым старэйшы курс, тым меней важна першая кампанента і тым болей другая.
% І ўсе заяўленыя пісьмовыя мэты павінны быць увязаны з гэтымі архімэтамі.
% А у тэксце што тут напісан, гэтага пакуль няма.
Казалось бы, кому в наше время может прийти в голову
идея создания собственного физического движка? Это далеко не
тривиальная задача, которую давно уже решили за вас
разработчики $Bullet$, $PhysX$, $ODE$, $Newton$ и других известных
библиотек симуляции динамики твердых тел. А дело вот в чем.
\\

Поскольку физика – это достаточно сложная предметная
область, требующая специальных знаний, к этим библиотекам
традиционно сложилось отношение как к «черным ящикам».

Пользователь не знает и не пытается понять все хитроумные
алгоритмы и тонкости внутренней архитектуры физического
движка, он лишь использует его внешний API. При этом он
может прекрасно разбираться в 3D-графике и линейной
алгебре, но физика для него так и останется «terra incognita»,
если он будет полагаться на готовые решения.
Знать физику с точки зрения пользователя физического
движка и с точки зрения его разработчика – это две совершенно
разные вещи.
\\

% Як ужо пісаў у натуральна-навуковых працах, нават "мы" не
% выкарыстоўваецца, і зноў тут эмоцыі і метафары. Вось у гэтым абзацы
% з'яўляюцца нейкі намёкі на сувязь з архімэтай, але трэба іх добра і нейтральна 
% выразіць
Изучение физики поможет нам лучше понять то,
как работает практически любой физический движок, перестать
относиться к физике как к некой «магии», научиться
использовать готовые решения более грамотно и эффективно.

% А вось тут у вас з'яўляецца вельмі цікавая думка.
% (Канешне, зноў яна напісана не адпаведным стылем).
% Тым не менш, калі мы крытыкуем існуючыя рашэнні за тое, што
% яны "камбайны", то варта падумаць, над тым ці не можам мы арганізаваць
% "наш" двіжок модульна, кшталту канвеераў і іншых рэчаў *nix'сітэмаў?
% і тады у нас з'яўлецца пэўная навізна, і адпаведна актуальнасць.
К тому же, абсолютное большинство движков перегружены 
функциональностью и представляют собой, фактически,
комбайны с собственной системой проверки столкновении,
громадной математической библиотекой (векторы, матрицы,
кватернионы), реализацией геометрических тел и еще Бог знает
чего еще. 

\section{Почему использование сторонней физики не оправданно?} 
\

Во многих случаях использование сторонней
физики как минимум неоправданно – например, если вы
создаете бильярд или боулинг, то вряд ли вам понадобятся
архисложный солвер, поддержка всех мыслимых типов
сочленений и неконвексной геометрии...%у рускай мове няма такога слова

Тем не менее, даже простой физический движок
реализовать весьма нелегко. В этой работе мы решили исправить эту
ситуацию и раскрыть ключевые моменты создания игровой физики.

\section{Цели и задачи исследования} 
\

% У навуковых працах (прынамсі на постсавецкай прасторы) мэта і задачы
% азначаюць вельмі канкрэтную рэч: мэта, гэта мэта усяго даследвання, 
% і для дасягненння гэтай мэты, неабходна вырашыць адпаведны шэраг задач.
В качестве главной задачи данного исследования, мы ставим цель
изучения законов физики твёрдого тела,  именно:

% тут трэба яшчэ велім падумаць над фармулёўкамі, бо зараз гэта выглядае так,
% нібыты вы фізік і збіраецеся праводзіць фізічныя даследванні (ды яшчэ на даўно 
% даследваную тэму. Вельмі веагодна што гэта стане адной задачай.
\begin{enumerate}
\item изучение законов изменения импульса, скорости и ускорения; \\%ізноў пазастылевае фарматаванне
\item изучение законов изменения тензора инерции, угловой скорости
и углового ускорения; \\ 
\item изучение законов изменения позиции твёрдого тела в пространстве
и времени, а также изменение и его ориентации; \\
\item изучение законов и правил взаимодействия двух твёрдых тел; \\
\end{enumerate}

% А вось гэта ужо іншая задача, таксама і ў наступным абзацы.
Сопутствующей целью будет просчет пересечений элементарных 
геометрических тел - поскольку любое физическое тело обладает
геометрической формой, которую нельзя игнорировать при имитации 
физических процессов.

Так же будут рассмотрены основные способы интегрирования
дифференциальных уравнений в рамках дисциплины численных методов.

\section{Основные литературные и информационные источники.}
\

% Ніколі не спускайцеся на узровень апраўдвання. Няма такой літаратуры ці бумажных крыніц
% пішыце якія ёсць. Дарэчы вы навуковыя артыкулы, на нашыя тэмы шукалі? А абавязкова трэба.
Поскольку большинство физических движков являются проприетарными
и совершенно закрытыми, найти хорошую литературу, а тем более
на русском языке, не представляется возможным. В связи с этим основным
источником информации стала всемирная сеть Интернет. Большая часть 
информации, представленной в данной работе, была получена путем изучения
и анализирования форумов и электронных журналов, посвященных
``игростроению``и игровой физике.

Из них следует отметить интернет-форум разработчиков игр gamedev.ru,
а так же свободнораспространяющийся электронный журнал ``FPS``.
\\

% Зноў атрымаліся нейкія апраўдванні, ніяк не тлумачыцца чаму абрана мова D
% у 20 папулярных моваў 20 моў :), чаму вы не пішыце гэты двіжок на джаваскрыпт,
% ці пітоне? Пра код увогуле можна шмат не пісаць, бо код праграм ніколі не змешчаецца у
% асноўным тэксце даследвання, а ў прылажэннях
В качестве референтного языка используется язык программирования D,
на момент написания данной работы, язык входил в 20 популярнейших языков
программирования. Несмотря на то, что данный язык программирования 
малоизвестен в широких кругах, представленный код будет понятен аналогичным 
для других родственных языков из семейства ``фигурных скобок``.



\end{document}