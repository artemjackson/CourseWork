\chapter{Предварительные замечания}
\section{Положения в современной физике}
\

То, что понимается под словом «физика» в приложениях
реального времени, практически полностью относится к
классической механике Ньютона – несмотря на то, что сама
физика как наука за последние столетия сделала громадный
шаг вперед, и ньютоновские законы уже не считаются
всеобъемлющими. Например, они неприменимы для объектов
со скоростями, близкими к скорости света (поведение таких
объектов описывает специальная теория относительности) или
для очень маленьких систем (таких, как атомы и элементарные
частицы – тут действуют законы квантовой механики). Область
применения ньютоновской механики – системы со скоростями
много меньше скорости света и размерами, значительно
превышающими размеры атомов и молекул. Законы Ньютона –
это следствие СТО для предельного случая 
\begin{equation}
v << c
\end{equation}

Необходимо заметить, что до сих пор не создана «теория всего»,
которая согласовала бы теорию относительности и квантовую
механику – это одна из главных нерешенных проблем
современной физики.
\\

\section{Пару слов об игровой физике}
\

Игровая физика, как правило, реализует раздел
механики, именуемый динамикой – то есть, движение тел с
учетом причин, вызывающих движение. Эти причины могут
включать различные силы (например, силу тяжести),
воздействие одних тел на другие (столкновения), воздействие
на тела различных ограничений степеней свободы (которые
иногда называют сочленениями или джоинтами).
